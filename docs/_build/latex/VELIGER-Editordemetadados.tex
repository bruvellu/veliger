% Generated by Sphinx.
\documentclass[letterpaper,10pt,portuguese]{manual}
\usepackage[utf8]{inputenc}
\usepackage[T1]{fontenc}
\usepackage{babel}
\usepackage{times}
\usepackage[Sonny]{fncychap}
\usepackage{longtable}
\usepackage{sphinx}


\title{VELIGER - Editor de metadados Documentation}
\date{24/05/2010}
\release{0.6.0b}
\author{Bruno C. Vellutini}
\newcommand{\sphinxlogo}{}
\renewcommand{\releasename}{Versão}
\makeindex
\makemodindex

\makeatletter
\def\PYG@reset{\let\PYG@it=\relax \let\PYG@bf=\relax%
    \let\PYG@ul=\relax \let\PYG@tc=\relax%
    \let\PYG@bc=\relax \let\PYG@ff=\relax}
\def\PYG@tok#1{\csname PYG@tok@#1\endcsname}
\def\PYG@toks#1+{\ifx\relax#1\empty\else%
    \PYG@tok{#1}\expandafter\PYG@toks\fi}
\def\PYG@do#1{\PYG@bc{\PYG@tc{\PYG@ul{%
    \PYG@it{\PYG@bf{\PYG@ff{#1}}}}}}}
\def\PYG#1#2{\PYG@reset\PYG@toks#1+\relax+\PYG@do{#2}}

\def\PYG@tok@gd{\def\PYG@tc##1{\textcolor[rgb]{0.63,0.00,0.00}{##1}}}
\def\PYG@tok@gu{\let\PYG@bf=\textbf\def\PYG@tc##1{\textcolor[rgb]{0.50,0.00,0.50}{##1}}}
\def\PYG@tok@gt{\def\PYG@tc##1{\textcolor[rgb]{0.00,0.25,0.82}{##1}}}
\def\PYG@tok@gs{\let\PYG@bf=\textbf}
\def\PYG@tok@gr{\def\PYG@tc##1{\textcolor[rgb]{1.00,0.00,0.00}{##1}}}
\def\PYG@tok@cm{\let\PYG@it=\textit\def\PYG@tc##1{\textcolor[rgb]{0.25,0.50,0.56}{##1}}}
\def\PYG@tok@vg{\def\PYG@tc##1{\textcolor[rgb]{0.73,0.38,0.84}{##1}}}
\def\PYG@tok@m{\def\PYG@tc##1{\textcolor[rgb]{0.13,0.50,0.31}{##1}}}
\def\PYG@tok@mh{\def\PYG@tc##1{\textcolor[rgb]{0.13,0.50,0.31}{##1}}}
\def\PYG@tok@cs{\def\PYG@tc##1{\textcolor[rgb]{0.25,0.50,0.56}{##1}}\def\PYG@bc##1{\colorbox[rgb]{1.00,0.94,0.94}{##1}}}
\def\PYG@tok@ge{\let\PYG@it=\textit}
\def\PYG@tok@vc{\def\PYG@tc##1{\textcolor[rgb]{0.73,0.38,0.84}{##1}}}
\def\PYG@tok@il{\def\PYG@tc##1{\textcolor[rgb]{0.13,0.50,0.31}{##1}}}
\def\PYG@tok@go{\def\PYG@tc##1{\textcolor[rgb]{0.19,0.19,0.19}{##1}}}
\def\PYG@tok@cp{\def\PYG@tc##1{\textcolor[rgb]{0.00,0.44,0.13}{##1}}}
\def\PYG@tok@gi{\def\PYG@tc##1{\textcolor[rgb]{0.00,0.63,0.00}{##1}}}
\def\PYG@tok@gh{\let\PYG@bf=\textbf\def\PYG@tc##1{\textcolor[rgb]{0.00,0.00,0.50}{##1}}}
\def\PYG@tok@ni{\let\PYG@bf=\textbf\def\PYG@tc##1{\textcolor[rgb]{0.84,0.33,0.22}{##1}}}
\def\PYG@tok@nl{\let\PYG@bf=\textbf\def\PYG@tc##1{\textcolor[rgb]{0.00,0.13,0.44}{##1}}}
\def\PYG@tok@nn{\let\PYG@bf=\textbf\def\PYG@tc##1{\textcolor[rgb]{0.05,0.52,0.71}{##1}}}
\def\PYG@tok@no{\def\PYG@tc##1{\textcolor[rgb]{0.38,0.68,0.84}{##1}}}
\def\PYG@tok@na{\def\PYG@tc##1{\textcolor[rgb]{0.25,0.44,0.63}{##1}}}
\def\PYG@tok@nb{\def\PYG@tc##1{\textcolor[rgb]{0.00,0.44,0.13}{##1}}}
\def\PYG@tok@nc{\let\PYG@bf=\textbf\def\PYG@tc##1{\textcolor[rgb]{0.05,0.52,0.71}{##1}}}
\def\PYG@tok@nd{\let\PYG@bf=\textbf\def\PYG@tc##1{\textcolor[rgb]{0.33,0.33,0.33}{##1}}}
\def\PYG@tok@ne{\def\PYG@tc##1{\textcolor[rgb]{0.00,0.44,0.13}{##1}}}
\def\PYG@tok@nf{\def\PYG@tc##1{\textcolor[rgb]{0.02,0.16,0.49}{##1}}}
\def\PYG@tok@si{\let\PYG@it=\textit\def\PYG@tc##1{\textcolor[rgb]{0.44,0.63,0.82}{##1}}}
\def\PYG@tok@s2{\def\PYG@tc##1{\textcolor[rgb]{0.25,0.44,0.63}{##1}}}
\def\PYG@tok@vi{\def\PYG@tc##1{\textcolor[rgb]{0.73,0.38,0.84}{##1}}}
\def\PYG@tok@nt{\let\PYG@bf=\textbf\def\PYG@tc##1{\textcolor[rgb]{0.02,0.16,0.45}{##1}}}
\def\PYG@tok@nv{\def\PYG@tc##1{\textcolor[rgb]{0.73,0.38,0.84}{##1}}}
\def\PYG@tok@s1{\def\PYG@tc##1{\textcolor[rgb]{0.25,0.44,0.63}{##1}}}
\def\PYG@tok@gp{\let\PYG@bf=\textbf\def\PYG@tc##1{\textcolor[rgb]{0.78,0.36,0.04}{##1}}}
\def\PYG@tok@sh{\def\PYG@tc##1{\textcolor[rgb]{0.25,0.44,0.63}{##1}}}
\def\PYG@tok@ow{\let\PYG@bf=\textbf\def\PYG@tc##1{\textcolor[rgb]{0.00,0.44,0.13}{##1}}}
\def\PYG@tok@sx{\def\PYG@tc##1{\textcolor[rgb]{0.78,0.36,0.04}{##1}}}
\def\PYG@tok@bp{\def\PYG@tc##1{\textcolor[rgb]{0.00,0.44,0.13}{##1}}}
\def\PYG@tok@c1{\let\PYG@it=\textit\def\PYG@tc##1{\textcolor[rgb]{0.25,0.50,0.56}{##1}}}
\def\PYG@tok@kc{\let\PYG@bf=\textbf\def\PYG@tc##1{\textcolor[rgb]{0.00,0.44,0.13}{##1}}}
\def\PYG@tok@c{\let\PYG@it=\textit\def\PYG@tc##1{\textcolor[rgb]{0.25,0.50,0.56}{##1}}}
\def\PYG@tok@mf{\def\PYG@tc##1{\textcolor[rgb]{0.13,0.50,0.31}{##1}}}
\def\PYG@tok@err{\def\PYG@bc##1{\fcolorbox[rgb]{1.00,0.00,0.00}{1,1,1}{##1}}}
\def\PYG@tok@kd{\let\PYG@bf=\textbf\def\PYG@tc##1{\textcolor[rgb]{0.00,0.44,0.13}{##1}}}
\def\PYG@tok@ss{\def\PYG@tc##1{\textcolor[rgb]{0.32,0.47,0.09}{##1}}}
\def\PYG@tok@sr{\def\PYG@tc##1{\textcolor[rgb]{0.14,0.33,0.53}{##1}}}
\def\PYG@tok@mo{\def\PYG@tc##1{\textcolor[rgb]{0.13,0.50,0.31}{##1}}}
\def\PYG@tok@mi{\def\PYG@tc##1{\textcolor[rgb]{0.13,0.50,0.31}{##1}}}
\def\PYG@tok@kn{\let\PYG@bf=\textbf\def\PYG@tc##1{\textcolor[rgb]{0.00,0.44,0.13}{##1}}}
\def\PYG@tok@o{\def\PYG@tc##1{\textcolor[rgb]{0.40,0.40,0.40}{##1}}}
\def\PYG@tok@kr{\let\PYG@bf=\textbf\def\PYG@tc##1{\textcolor[rgb]{0.00,0.44,0.13}{##1}}}
\def\PYG@tok@s{\def\PYG@tc##1{\textcolor[rgb]{0.25,0.44,0.63}{##1}}}
\def\PYG@tok@kp{\def\PYG@tc##1{\textcolor[rgb]{0.00,0.44,0.13}{##1}}}
\def\PYG@tok@w{\def\PYG@tc##1{\textcolor[rgb]{0.73,0.73,0.73}{##1}}}
\def\PYG@tok@kt{\def\PYG@tc##1{\textcolor[rgb]{0.56,0.13,0.00}{##1}}}
\def\PYG@tok@sc{\def\PYG@tc##1{\textcolor[rgb]{0.25,0.44,0.63}{##1}}}
\def\PYG@tok@sb{\def\PYG@tc##1{\textcolor[rgb]{0.25,0.44,0.63}{##1}}}
\def\PYG@tok@k{\let\PYG@bf=\textbf\def\PYG@tc##1{\textcolor[rgb]{0.00,0.44,0.13}{##1}}}
\def\PYG@tok@se{\let\PYG@bf=\textbf\def\PYG@tc##1{\textcolor[rgb]{0.25,0.44,0.63}{##1}}}
\def\PYG@tok@sd{\let\PYG@it=\textit\def\PYG@tc##1{\textcolor[rgb]{0.25,0.44,0.63}{##1}}}

\def\PYGZbs{\char`\\}
\def\PYGZus{\char`\_}
\def\PYGZob{\char`\{}
\def\PYGZcb{\char`\}}
\def\PYGZca{\char`\^}
% for compatibility with earlier versions
\def\PYGZat{@}
\def\PYGZlb{[}
\def\PYGZrb{]}
\makeatother

\begin{document}
\shorthandoff{"}
\maketitle
\tableofcontents
\hypertarget{--doc-index}{}


VÉLIGER é o editor de metadados do banco de imagens do Centro de Biologia
Marinha da Universidade de São Paulo (CEBIMar-USP).

Este programa abre imagens JPG, lê seus metadados (IPTC) e fornece uma
interface para editar estas informações. Os campos foram adaptados para o
propósito do banco, que é divulgar imagens relacionadas com biologia marinha.

Os campos editáveis são:
\begin{description}
\item[IPTC] \leavevmode
título, legenda, marcadores, táxon, espécie, especialista, autor, direitos,
tamanho, local, cidade, estado e país.

\item[EXIF] \leavevmode
Geolocalização.

\end{description}


\chapter{Guia rápido}

Instale o programa seguindo estas instruções (link) e inicie-o.

Ao importar imagens o programa irá ler os metadados de cada uma e criar as
respectivas entradas na tabela principal. Selecione uma linha da tabela para
visualizar o thumbnail e os metadados contidos na imagem.

Para editar clique 2x na célula da tabela ou clique e aperte F2; após digitar
aperte \emph{Enter} e o valor estará salvo no programa (mas ainda não foi gravado na
imagem).

Outra maneira de editar é usando os campos de edição. Após selecionar uma
entrada na tabela os campos de edição serão preenchidos com os metadados da
imagem. Para editar simplesmente escreva as informações nos campos desejados e
aperte \textbf{ctrl+s} \footnote{
ou use o ícone para salvar no menu de ferramentas.
} antes de mudar para outra imagem. Caso você mude a linha
selecionada antes de salvar as alterações serão perdidas.

\begin{notice}{note}{Nota:}
Lembre-se que salvar com ctrl+s, ou editando direto na tabela, não
grava as alterações no arquivo da imagem. Os novos metadados ficam salvos
apenas no programa, mas de maneira persistente (não é perdido após
reiniciá-lo).
\end{notice}

Após a alteração dos metadados a entrada é adicionada à uma lista que mostra os
arquivos cujos metadados foram modificados, mas ainda não foram gravados no
arquivo (ou seja, o arquivo ainda está com os metadados originais). Para gravar
os metadados na imagem utilize o atalho \textbf{ctrl+shift+s} \footnote{
ou clique em gravar na doca de modificadas.
}. Se tudo correr
bem na gravação os novos metadados estarão embebidos na imagem.

\begin{notice}{note}{Nota:}
A codificação utilizada para gravar os metadados é UTF-8. Veja mais
informações sobre a codificação dos metadados abaixo.
\end{notice}

Para editar vários valores de uma coluna, selecione a célula de uma entrada e
arraste até outra entrada na mesma coluna (ou clique e segure \textbf{shift} ou
\textbf{ctrl} para selecionar outras); com as entradas selecionadas aperte F2 e ele
vai abrir o campinho de edição na tabela; escreva e aperte \emph{Enter} e os valores
serão copiados para as entradas selecionadas (apenas para a coluna escolhida).

Para alterar todos os metadados de várias entradas de uma vez selecione as
entradas que serão modificadas e preencha os campos de edição. Ao apertar
\textbf{ctrl+s} todas as entradas selecionadas terão seus metadados
sobrescritos pelo que você tiver inserido nos campos de edição.

Também é possível copiar os metadados de uma entrada e colá-los em outra(s).
Para tal, basta selecionar uma entrada e apertar \textbf{ctrl+c} para copiar,
selecionar as entradas cujos metadados serão sobrescritos e apertar \textbf{ctrl+v}.

\begin{notice}{note}{Nota:}
Lembre-se que ao salvar os metadados (\textbf{ctrl+s}) todos os metadados das
entradas selecionadas serão \textbf{sobrescritos}, mesmo você tendo modificado
apenas um dos campos (a legenda, por exemplo). O mesmo ocorrerá ao colar os
metadados copiados de uma entrada em outras usando \textbf{ctrl+c}/\textbf{ctrl+v}. O
resultado destas duas funções são entradas cujos metadados são idênticos.
\end{notice}

Ao fechar o programa ele lembra a tabela e suas modificações, portanto pode
fechar sem apertar \textbf{Ctrl+Shift+S} que as modificações continuaram presentes
quando programa for aberto novamente.


\section{Geolocalização}

Ao clicar numa entrada, se a doca da geolocalização estiver aberta, o mapa e as
coordenadas gravadas na imagem serão carregadas. Para mudar a localização da
imagem simplesmente arraste o marcador no mapa até a nova posição, aperte
o botão \textbf{Atualizar} para carregar a nova localização no editor e aperte
\textbf{Gravar} para salvar as alterações na imagem.

Se múltiplas imagens estiverem selecionadas, a geolocalização será gravada em
todas.

\begin{notice}{note}{Nota:}
É necessário apertar \textbf{Atualizar} após arrastar. As novas coordenadas não
são atualizadas automaticamente (ainda).
\end{notice}

Imagens sem coordenadas gravadas no EXIF mostrarão um mapa sem marcador. Para
selecionar um local clique com o \textbf{botão direito} do mouse no mapa \footnote{
clicar diversas vezes irá criar diversos marcadores, mas apenas o último
criado ou arrastado que terá suas coordenadas lidas.
}. O zoom
será aumentado para que você possa refinar a posição arrastando o marcador.
Lembre-se de atualizar antes de gravar.

\begin{notice}{note}{Nota:}
Existe um bug no API do Google Maps ou no WebKit que faz com que o marcador
seja colocado na posição errada. Isso ocorre quando o usuário arrasta o
mapa ou aumenta o zoom, mudando a posição central inicial, antes de criar o
marcador. Por este motivo, ao criar um novo marcador tente clicar na região
desejada primeiro para evitar contratempos. Após criá-lo não haverá problemas
para arrastar até a posição específica.
\end{notice}


\section{Conversão da codificação}

Se você observar erros na codificação de caracteres especiais (ç, à, á, ã, ê)
dos metadados após importar imagens, o programa que inseriu estes dados deve
ter utilizado o padrão Latin-1 (ISO 8859-1). Para corrigir a codificação para o
padrão preferido atualmente, o UTF-8, utilize a função de conversão no menu
Editar. Essa alteração é gravada na imagem, logo, tenha backup sempre, pois se
você voltar a editar esta imagem com outro programa pode ser que ele não
reconheça o UTF-8 corretamente (acho difícil, mas pode acontecer).


\section{Docas}

A janela principal do \textbf{VÉLIGER} contém 4 docas: editor dos metadados, editor
da geolocalização, miniatura da imagem e lista de entradas modificadas. Estas docas
tem um posicionamento padrão, mas podem ser modificados pelo usuário; são arrastáveis,
móveis e destacáveis.


\chapter{Atalhos de teclado}

Principais atalhos de teclado disponíveis:

\begin{tabulary}{\textwidth}{|L|L|}
\hline
\textbf{
Atalho
} & \textbf{
Função
}\\
\hline

ctrl+o
 & 
Importar arquivo(s) .jpg
\\

ctrl+d
 & 
Importar recursivamente todo o conteúdo de uma pasta
\\

ctrl+c
 & 
Copiar metadados da entrada selecionada
\\

ctrl+v
 & 
Colar metadados na(s) entrada(s) selecionada(s)
\\

ctrl+s
 & 
Salvar os metadados nos campos de edição para a tabela
\\

ctrl+shift+s
 & 
Gravar os metadados alterados nas respectivas imagens
\\

alt+{[}a-z{]}
 & 
Focar o cursor nos menus ou campos de edição
(e.g., alt+p colocará o cursor no campo \textbf{País})
\\

shift+e
 & 
Mostrar/esconder a doca com campos de edição
\\

shift+g
 & 
Mostrar/esconder a doca com geolocalização
\\

shift+t
 & 
Mostrar/esconder a doca com miniatura da imagem
\\

shift+u
 & 
Mostrar/esconder a doca com lista de entradas modificadas
\\

ctrl+q
 & 
Sair do programa
\\
\hline
\end{tabulary}


Algumas funções mais drásticas como limpar a tabela principal e converter a
codificação dos caracteres não tem atalhos para evitar acidentes.


\chapter{Tópicos}

\resetcurrentobjects
\hypertarget{--doc-fullapi}{}

\section{Descrição das classes e funções do VÉLIGER}
\index{veliger (módulo)}
\hypertarget{module-veliger}{}
\declaremodule[veliger]{}{veliger}
\modulesynopsis{}
Editor de metadados do banco de imagens do CEBIMar-USP.

Este programa abre imagens JPG, lê seus metadados (IPTC) e fornece uma
interface para editar estas informações. Os campos foram adaptados para o
propósito do banco, que é divulgar imagens com conteúdo biológico.

Campos editáveis: título, legenda, marcadores, táxon, espécie, especialista,
autor, direitos, tamanho, local, cidade, estado e país.

Centro de Biologia Marinha da Universidade de São Paulo.
\index{AboutDialog (classe em veliger)}

\hypertarget{veliger.AboutDialog}{}\begin{classdesc}{AboutDialog}{parent}
Janela com informações sobre o programa.
\end{classdesc}
\index{AutoModels (classe em veliger)}

\hypertarget{veliger.AutoModels}{}\begin{classdesc}{AutoModels}{list}
Cria modelos para autocompletar campos de edição.
\end{classdesc}
\index{CompleterLineEdit (classe em veliger)}

\hypertarget{veliger.CompleterLineEdit}{}\begin{classdesc}{CompleterLineEdit}{*args}
Editor especial para marcadores.

Adaptado de John Schember:
john.nachtimwald.com/2009/07/04/qcompleter-and-comma-separated-tags/
\end{classdesc}
\index{DockEditor (classe em veliger)}

\hypertarget{veliger.DockEditor}{}\begin{classdesc}{DockEditor}{parent}
Dock com campos para edição dos metadados.
\index{autolistgen() (método veliger.DockEditor)}

\hypertarget{veliger.DockEditor.autolistgen}{}\begin{methoddesc}{autolistgen}{models}
Gera autocompletadores dos campos.
\end{methoddesc}
\index{savedata() (método veliger.DockEditor)}

\hypertarget{veliger.DockEditor.savedata}{}\begin{methoddesc}{savedata}{}
Salva valores dos campos para a tabela.
\end{methoddesc}
\index{setcurrent() (método veliger.DockEditor)}

\hypertarget{veliger.DockEditor.setcurrent}{}\begin{methoddesc}{setcurrent}{values}
Atualiza campos de edição quando entrada é selecionada na tabela.
\end{methoddesc}
\index{setsingle() (método veliger.DockEditor)}

\hypertarget{veliger.DockEditor.setsingle}{}\begin{methoddesc}{setsingle}{index, value, oldvalue}
Atualiza campo de edição correspondente quando dado é alterado.
\end{methoddesc}
\end{classdesc}
\index{DockGeo (classe em veliger)}

\hypertarget{veliger.DockGeo}{}\begin{classdesc}{DockGeo}{parent}
Dock para editar a geolocalização da imagem.
\index{geodict() (método veliger.DockGeo)}

\hypertarget{veliger.DockGeo.geodict}{}\begin{methoddesc}{geodict}{string}
Extrai coordenadas da string do editor.
\end{methoddesc}
\index{get\_decimal() (método veliger.DockGeo)}

\hypertarget{veliger.DockGeo.get\_decimal}{}\begin{methoddesc}{get\_decimal}{ref, deg, min, sec}
Descobre o valor decimal das coordenadas.
\end{methoddesc}
\index{get\_exif() (método veliger.DockGeo)}

\hypertarget{veliger.DockGeo.get\_exif}{}\begin{methoddesc}{get\_exif}{filepath}
Extrai o exif da imagem selecionada usando o pyexiv2 0.1.3.
\end{methoddesc}
\index{load\_geocode() (método veliger.DockGeo)}

\hypertarget{veliger.DockGeo.load\_geocode}{}\begin{methoddesc}{load\_geocode}{gps}
Pega coordenadas e chama mapa com as variáveis correspondetes.
\end{methoddesc}
\index{newgps() (método veliger.DockGeo)}

\hypertarget{veliger.DockGeo.newgps}{}\begin{methoddesc}{newgps}{}
Pega as novas coordenadas do editor.
\end{methoddesc}
\index{resolve() (método veliger.DockGeo)}

\hypertarget{veliger.DockGeo.resolve}{}\begin{methoddesc}{resolve}{frac}
Resolve a fração das coordenadas para int.

Por padrão os valores do exif são guardados como frações. Por isso é
necessário converter.
\end{methoddesc}
\index{setcurrent() (método veliger.DockGeo)}

\hypertarget{veliger.DockGeo.setcurrent}{}\begin{methoddesc}{setcurrent}{values}
Mostra geolocalização da imagem selecionada.
\end{methoddesc}
\index{setdms() (método veliger.DockGeo)}

\hypertarget{veliger.DockGeo.setdms}{}\begin{methoddesc}{setdms}{dms}
Atualiza as coordenadas do editor.
\end{methoddesc}
\index{state() (método veliger.DockGeo)}

\hypertarget{veliger.DockGeo.state}{}\begin{methoddesc}{state}{visible}
Relata se aba está visível e/ou selecionada.

Captura sinal emitido pelo mainWidget.

Por não distinguir entre visível e aba selecionada não funciona em
algumas situações.
\end{methoddesc}
\index{un\_decimal() (método veliger.DockGeo)}

\hypertarget{veliger.DockGeo.un\_decimal}{}\begin{methoddesc}{un\_decimal}{lat, long}
Converte o valor decimal das coordenadas.

Retorna dicionário com referência cardinal, graus, minutos e segundos.
\end{methoddesc}
\index{update\_geo() (método veliger.DockGeo)}

\hypertarget{veliger.DockGeo.update\_geo}{}\begin{methoddesc}{update\_geo}{}
Captura as coordenadas do marcador para atualizar o editor.
\end{methoddesc}
\index{write\_geo() (método veliger.DockGeo)}

\hypertarget{veliger.DockGeo.write\_geo}{}\begin{methoddesc}{write\_geo}{}
Grava novas coordenadas nas imagens selecionadas.
\end{methoddesc}
\index{write\_html() (método veliger.DockGeo)}

\hypertarget{veliger.DockGeo.write\_html}{}\begin{methoddesc}{write\_html}{unset=0, lat=0.0, long=0.0, zoom=5}
Carrega código HTML da QWebView com mapa do Google Maps.

Usando o API V3.
\end{methoddesc}
\end{classdesc}
\index{DockThumb (classe em veliger)}

\hypertarget{veliger.DockThumb}{}\begin{classdesc}{DockThumb}{}
Dock para mostrar o thumbnail da imagem selecionada.
\index{pixmapcache() (método veliger.DockThumb)}

\hypertarget{veliger.DockThumb.pixmapcache}{}\begin{methoddesc}{pixmapcache}{filepath}
Cria cache para thumbnail.
\end{methoddesc}
\index{resizeEvent() (método veliger.DockThumb)}

\hypertarget{veliger.DockThumb.resizeEvent}{}\begin{methoddesc}{resizeEvent}{event}
Lida com redimensionamento do thumbnail.
\end{methoddesc}
\index{setcurrent() (método veliger.DockThumb)}

\hypertarget{veliger.DockThumb.setcurrent}{}\begin{methoddesc}{setcurrent}{values}
Mostra thumbnail, nome e data de modificação da imagem.

Captura sinal com valores, tenta achar imagem no cache e exibe
informações.
\end{methoddesc}
\index{updateThumb() (método veliger.DockThumb)}

\hypertarget{veliger.DockThumb.updateThumb}{}\begin{methoddesc}{updateThumb}{}
Atualiza thumbnail.
\end{methoddesc}
\end{classdesc}
\index{DockUnsaved (classe em veliger)}

\hypertarget{veliger.DockUnsaved}{}\begin{classdesc}{DockUnsaved}{}
Exibe lista com imagens modificadas.

Utiliza dados do modelo em lista. Qualquer imagem modificada será
adicionada à lista. Seleção na lista seleciona entrada na tabela. Gravar
salva metadados de cada ítem da lista nas respectivas imagens.
\index{clearlist() (método veliger.DockUnsaved)}

\hypertarget{veliger.DockUnsaved.clearlist}{}\begin{methoddesc}{clearlist}{}
Remove todas as entradas da lista.
\end{methoddesc}
\index{delentry() (método veliger.DockUnsaved)}

\hypertarget{veliger.DockUnsaved.delentry}{}\begin{methoddesc}{delentry}{filename}
Remove entrada da lista.
\end{methoddesc}
\index{insertentry() (método veliger.DockUnsaved)}

\hypertarget{veliger.DockUnsaved.insertentry}{}\begin{methoddesc}{insertentry}{index, value, oldvalue}
Insere entrada na lista.

Checa se a modificação não foi nula (valor atual == valor anterior) e
se a entrada é duplicada.
\end{methoddesc}
\index{matchfinder() (método veliger.DockUnsaved)}

\hypertarget{veliger.DockUnsaved.matchfinder}{}\begin{methoddesc}{matchfinder}{candidate}
Buscador de duplicatas.
\end{methoddesc}
\index{resizeEvent() (método veliger.DockUnsaved)}

\hypertarget{veliger.DockUnsaved.resizeEvent}{}\begin{methoddesc}{resizeEvent}{event}
Lida com redimensionamentos.
\end{methoddesc}
\index{sync\_setselection() (método veliger.DockUnsaved)}

\hypertarget{veliger.DockUnsaved.sync\_setselection}{}\begin{methoddesc}{sync\_setselection}{selected, deselected}
Sincroniza seleção da tabela com a seleção da lista.
\end{methoddesc}
\index{writeselected() (método veliger.DockUnsaved)}

\hypertarget{veliger.DockUnsaved.writeselected}{}\begin{methoddesc}{writeselected}{}
Emite sinal para gravar metadados na imagem.
\end{methoddesc}
\end{classdesc}
\index{EditCompletion (classe em veliger)}

\hypertarget{veliger.EditCompletion}{}\begin{classdesc}{EditCompletion}{parent}
Editor dos valores para autocompletar campos de edição.
\index{buildview() (método veliger.EditCompletion)}

\hypertarget{veliger.EditCompletion.buildview}{}\begin{methoddesc}{buildview}{modellist}
Gera view do modelo escolhido.
\end{methoddesc}
\index{insertrow() (método veliger.EditCompletion)}

\hypertarget{veliger.EditCompletion.insertrow}{}\begin{methoddesc}{insertrow}{}
Insere linha no modelo.
\end{methoddesc}
\index{populate() (método veliger.EditCompletion)}

\hypertarget{veliger.EditCompletion.populate}{}\begin{methoddesc}{populate}{}
Extrai valores das fotos e popula modelo.
\end{methoddesc}
\index{removerow() (método veliger.EditCompletion)}

\hypertarget{veliger.EditCompletion.removerow}{}\begin{methoddesc}{removerow}{}
Remove linha do modelo.
\end{methoddesc}
\end{classdesc}
\index{FlushFile (classe em veliger)}

\hypertarget{veliger.FlushFile}{}\begin{classdesc}{FlushFile}{f}
Tira o buffer do print.

Assim rola juntar prints diferentes na mesma linha. Só pra ficar
bonitinho... meio inútil.
\end{classdesc}
\index{InitPs (classe em veliger)}

\hypertarget{veliger.InitPs}{}\begin{classdesc}{InitPs}{}
Inicia variáveis e parâmetros globais do programa.
\end{classdesc}
\index{ListModel (classe em veliger)}

\hypertarget{veliger.ListModel}{}\begin{classdesc}{ListModel}{parent, list, *args}
Modelo com lista de imagens modificadas.
\index{data() (método veliger.ListModel)}

\hypertarget{veliger.ListModel.data}{}\begin{methoddesc}{data}{index, role}
Cria elementos da lista a partir dos dados.
\end{methoddesc}
\index{insert\_rows() (método veliger.ListModel)}

\hypertarget{veliger.ListModel.insert\_rows}{}\begin{methoddesc}{insert\_rows}{position, rows, parent, entry}
Insere linhas.
\end{methoddesc}
\index{remove\_rows() (método veliger.ListModel)}

\hypertarget{veliger.ListModel.remove\_rows}{}\begin{methoddesc}{remove\_rows}{position, rows, parent}
Remove linhas.
\end{methoddesc}
\index{rowCount() (método veliger.ListModel)}

\hypertarget{veliger.ListModel.rowCount}{}\begin{methoddesc}{rowCount}{parent}
Conta linhas.
\end{methoddesc}
\end{classdesc}
\index{MainCompleter (classe em veliger)}

\hypertarget{veliger.MainCompleter}{}\begin{classdesc}{MainCompleter}{model, parent}
Autocomplete principal.
\end{classdesc}
\index{MainTable (classe em veliger)}

\hypertarget{veliger.MainTable}{}\begin{classdesc}{MainTable}{datalist, header, *args}
Tabela principal com entradas.
\index{changecurrent() (método veliger.MainTable)}

\hypertarget{veliger.MainTable.changecurrent}{}\begin{methoddesc}{changecurrent}{current, previous}
Identifica a célula selecionada, extrai valores e envia para editor.

Os valores são enviados através de um sinal.
\end{methoddesc}
\index{editmultiple() (método veliger.MainTable)}

\hypertarget{veliger.MainTable.editmultiple}{}\begin{methoddesc}{editmultiple}{index, value, oldvalue}
Edita outras linhas selecionadas.
\end{methoddesc}
\index{emitlost() (método veliger.MainTable)}

\hypertarget{veliger.MainTable.emitlost}{}\begin{methoddesc}{emitlost}{filename}
Emite aviso para remover entrada da lista de modificados.
\end{methoddesc}
\index{emitsaved() (método veliger.MainTable)}

\hypertarget{veliger.MainTable.emitsaved}{}\begin{methoddesc}{emitsaved}{}
Emite aviso que os metadados foram gravados nos arquivos.
\end{methoddesc}
\index{outputrows() (método veliger.MainTable)}

\hypertarget{veliger.MainTable.outputrows}{}\begin{methoddesc}{outputrows}{toprow}
Identifica linhas dentro do campo de visão da tabela.
\end{methoddesc}
\index{resizecols() (método veliger.MainTable)}

\hypertarget{veliger.MainTable.resizecols}{}\begin{methoddesc}{resizecols}{index}
Ajusta largura das colunas da tabela.
\end{methoddesc}
\index{update\_selection() (método veliger.MainTable)}

\hypertarget{veliger.MainTable.update\_selection}{}\begin{methoddesc}{update\_selection}{selected, deselected}
Pega a entrada selecionada, extrai os valores envia para editor.

Os valores são enviados através de um sinal.
\end{methoddesc}
\end{classdesc}
\index{MainWindow (classe em veliger)}

\hypertarget{veliger.MainWindow}{}\begin{classdesc}{MainWindow}{}
Janela principal do programa.

Inicia as instâncias dos outros componentes e aguarda interação do usuário.
\index{cachetable() (método veliger.MainWindow)}

\hypertarget{veliger.MainWindow.cachetable}{}\begin{methoddesc}{cachetable}{}
Salva estado atual dos dados em arquivos externos.

Cria backup dos conteúdos da tabela e da lista de imagens modificadas.
\end{methoddesc}
\index{changeStatus() (método veliger.MainWindow)}

\hypertarget{veliger.MainWindow.changeStatus}{}\begin{methoddesc}{changeStatus}{status, duration=2000}
Muda a mensagem de status da janela principal.
\end{methoddesc}
\index{charconverter() (método veliger.MainWindow)}

\hypertarget{veliger.MainWindow.charconverter}{}\begin{methoddesc}{charconverter}{}
Converte codificação de Latin-1 para UTF-8.

Pega a seleção da lista de imagens modificadas e procura a linha
correspondente na tabela principal. Se o item não for encontrado o item
na lista é apagado.
\end{methoddesc}
\index{cleartable() (método veliger.MainWindow)}

\hypertarget{veliger.MainWindow.cleartable}{}\begin{methoddesc}{cleartable}{}
Remove todas as entradas da tabela.

Antes de deletar checa se existem imagens não-salvas na lista.
\end{methoddesc}
\index{closeEvent() (método veliger.MainWindow)}

\hypertarget{veliger.MainWindow.closeEvent}{}\begin{methoddesc}{closeEvent}{event}
O que fazer quando o programa for fechado.
\end{methoddesc}
\index{commitmeta() (método veliger.MainWindow)}

\hypertarget{veliger.MainWindow.commitmeta}{}\begin{methoddesc}{commitmeta}{entries}
Grava os metadados modificados na imagem.

Pega lista de imagens modificadas, procura entrada na tabela principal
e retorna os metadados. Chama função que gravará estes metadados na
imagem. Chama função que emitirá o sinal avisando a gravação foi
completada com sucesso.
\end{methoddesc}
\index{copydata() (método veliger.MainWindow)}

\hypertarget{veliger.MainWindow.copydata}{}\begin{methoddesc}{copydata}{}
Copia metadados da entrada selecionada.
\end{methoddesc}
\index{createmeta() (método veliger.MainWindow)}

\hypertarget{veliger.MainWindow.createmeta}{}\begin{methoddesc}{createmeta}{filepath, charset='utf-8'}
Define as variáveis extraídas dos metadados (IPTC) da imagem.

Usa a biblioteca do arquivo iptcinfo.py e retorna lista com valores.
\end{methoddesc}
\index{createthumbs() (método veliger.MainWindow)}

\hypertarget{veliger.MainWindow.createthumbs}{}\begin{methoddesc}{createthumbs}{filepath}
Cria thumbnails para as fotos novas.
\end{methoddesc}
\index{delcurrent() (método veliger.MainWindow)}

\hypertarget{veliger.MainWindow.delcurrent}{}\begin{methoddesc}{delcurrent}{}
Deleta a(s) entrada(s) selecionada(s) da tabela.

Verifica se a entrada a ser deletada está na lista de imagens
modificadas. Se estiver, chama janela para o usuário decidir se quer
apagar a entrada mesmo sem as modificações terem sido gravadas na
imagem. Caso a resposta seja positiva a entrada será apagada e retirada
da lista de imagens modificadas.
\end{methoddesc}
\index{exiftool() (método veliger.MainWindow)}

\hypertarget{veliger.MainWindow.exiftool}{}\begin{methoddesc}{exiftool}{values}
Grava os metadados no arquivo.

Usa subprocesso para chamar o exiftool, que gravará os metadados
modificados na imagem.

Função não utilizada, mas pode ser útil algum dia.
\end{methoddesc}
\index{imgfinder() (método veliger.MainWindow)}

\hypertarget{veliger.MainWindow.imgfinder}{}\begin{methoddesc}{imgfinder}{folder}
Busca recursivamente imagens na pasta selecionada.

É possível definir as extensões a serem procuradas. Quando um arquivo é
encontrado ele verifica se já está na tabela. Se não estiver, ele chama
a função para extrair os metadados e insere uma nova entrada.
\end{methoddesc}
\index{matchfinder() (método veliger.MainWindow)}

\hypertarget{veliger.MainWindow.matchfinder}{}\begin{methoddesc}{matchfinder}{candidate}
Verifica se entrada já está na tabela.

O candidato pode ser o nome do arquivo (string) ou a entrada
selecionada da tabela (lista). Retorna uma lista com duplicatas ou
lista vazia caso nenhuma seja encontrada.
\end{methoddesc}
\index{openabout\_dialog() (método veliger.MainWindow)}

\hypertarget{veliger.MainWindow.openabout\_dialog}{}\begin{methoddesc}{openabout\_dialog}{}
Abre janela sobre o programa.
\end{methoddesc}
\index{opendir\_dialog() (método veliger.MainWindow)}

\hypertarget{veliger.MainWindow.opendir\_dialog}{}\begin{methoddesc}{opendir\_dialog}{}
Abre janela para selecionar uma pasta.

Chama a função para varrer a pasta selecionada.
\end{methoddesc}
\index{openfile\_dialog() (método veliger.MainWindow)}

\hypertarget{veliger.MainWindow.openfile\_dialog}{}\begin{methoddesc}{openfile\_dialog}{}
Abre janela para escolher arquivos.

Para selecionar arquivo(s) terminados em .jpg.
\end{methoddesc}
\index{openmanual\_dialog() (método veliger.MainWindow)}

\hypertarget{veliger.MainWindow.openmanual\_dialog}{}\begin{methoddesc}{openmanual\_dialog}{}
Abre janela do manual de instruções.
\end{methoddesc}
\index{openpref\_dialog() (método veliger.MainWindow)}

\hypertarget{veliger.MainWindow.openpref\_dialog}{}\begin{methoddesc}{openpref\_dialog}{}
Abre janela de opções.
\end{methoddesc}
\index{pastedata() (método veliger.MainWindow)}

\hypertarget{veliger.MainWindow.pastedata}{}\begin{methoddesc}{pastedata}{}
Cola metadados na(s) entrada(s) selecionada(s).
\end{methoddesc}
\index{readsettings() (método veliger.MainWindow)}

\hypertarget{veliger.MainWindow.readsettings}{}\begin{methoddesc}{readsettings}{}
Lê o estado anterior do aplicativo durante a inicialização.
\end{methoddesc}
\index{setselection() (método veliger.MainWindow)}

\hypertarget{veliger.MainWindow.setselection}{}\begin{methoddesc}{setselection}{filename}
Sincroniza seleção entre lista e tabela principal.

Pega a seleção da lista de imagens modificadas e procura a linha
correspondente na tabela principal. Se o item não for encontrado o item
na lista é apagado.
\end{methoddesc}
\index{writemeta() (método veliger.MainWindow)}

\hypertarget{veliger.MainWindow.writemeta}{}\begin{methoddesc}{writemeta}{values}
Grava os metadados no arquivo.
\end{methoddesc}
\index{writesettings() (método veliger.MainWindow)}

\hypertarget{veliger.MainWindow.writesettings}{}\begin{methoddesc}{writesettings}{}
Salva estado atual do aplicativo.
\end{methoddesc}
\end{classdesc}
\index{ManualDialog (classe em veliger)}

\hypertarget{veliger.ManualDialog}{}\begin{classdesc}{ManualDialog}{parent}
Janela do manual de instruções.
\end{classdesc}
\index{PrefsDialog (classe em veliger)}

\hypertarget{veliger.PrefsDialog}{}\begin{classdesc}{PrefsDialog}{parent}
Janela de preferências.
\index{emitrebuild() (método veliger.PrefsDialog)}

\hypertarget{veliger.PrefsDialog.emitrebuild}{}\begin{methoddesc}{emitrebuild}{}
Emite sinal com modelos atualizados e ordenados.
\end{methoddesc}
\end{classdesc}
\index{PrefsGerais (classe em veliger)}

\hypertarget{veliger.PrefsGerais}{}\begin{classdesc}{PrefsGerais}{}
Opções gerais do programa.
\end{classdesc}
\index{TableModel (classe em veliger)}

\hypertarget{veliger.TableModel}{}\begin{classdesc}{TableModel}{parent, mydata, header, *args}
Modelo dos dados.
\index{columnCount() (método veliger.TableModel)}

\hypertarget{veliger.TableModel.columnCount}{}\begin{methoddesc}{columnCount}{parent}
Conta as colunas.
\end{methoddesc}
\index{data() (método veliger.TableModel)}

\hypertarget{veliger.TableModel.data}{}\begin{methoddesc}{data}{index, role}
Transforma dados brutos em elementos da tabela.
\end{methoddesc}
\index{flags() (método veliger.TableModel)}

\hypertarget{veliger.TableModel.flags}{}\begin{methoddesc}{flags}{index}
Indicadores do estado de cada ítem.
\end{methoddesc}
\index{headerData() (método veliger.TableModel)}

\hypertarget{veliger.TableModel.headerData}{}\begin{methoddesc}{headerData}{col, orientation, role}
Constrói cabeçalho da tabela.
\end{methoddesc}
\index{insert\_rows() (método veliger.TableModel)}

\hypertarget{veliger.TableModel.insert\_rows}{}\begin{methoddesc}{insert\_rows}{position, rows, parent, entry}
Insere entrada na tabela.
\end{methoddesc}
\index{remove\_rows() (método veliger.TableModel)}

\hypertarget{veliger.TableModel.remove\_rows}{}\begin{methoddesc}{remove\_rows}{position, rows, parent}
Remove entrada da tabela.
\end{methoddesc}
\index{rowCount() (método veliger.TableModel)}

\hypertarget{veliger.TableModel.rowCount}{}\begin{methoddesc}{rowCount}{parent}
Conta as linhas.
\end{methoddesc}
\index{setData() (método veliger.TableModel)}

\hypertarget{veliger.TableModel.setData}{}\begin{methoddesc}{setData}{index, value, role}
Salva alterações nos dados a partir da edição da tabela.
\end{methoddesc}
\index{setcolor() (método veliger.TableModel)}

\hypertarget{veliger.TableModel.setcolor}{}\begin{methoddesc}{setcolor}{index, role}
Pinta células da tabela.
\end{methoddesc}
\index{sort() (método veliger.TableModel)}

\hypertarget{veliger.TableModel.sort}{}\begin{methoddesc}{sort}{col, order}
Ordena entradas a partir de valores de determinada coluna
\end{methoddesc}
\end{classdesc}
\index{TagCompleter (classe em veliger)}

\hypertarget{veliger.TagCompleter}{}\begin{classdesc}{TagCompleter}{model, parent}
Completador de marcadores.

Adaptado de John Schember:
john.nachtimwald.com/2009/07/04/qcompleter-and-comma-separated-tags/
\end{classdesc}


\chapter{Índices e tabelas}
\begin{itemize}
\item {} 
\emph{Index}

\item {} 
\emph{Module Index}

\item {} 
\emph{Search Page}

\end{itemize}


\renewcommand{\indexname}{Índice do Módulo}
\printmodindex
\renewcommand{\indexname}{Índice}
\printindex
\end{document}
